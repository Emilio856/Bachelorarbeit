In dieser Arbeit wurden neun Architekturen und einige ihrer Varianten, also dreizehn faltende neuronale Netze bezüglich einer Regression der Picklänge kardiovaskulärer Implantate untersucht. Hierfür wurden zur Verfügung gestellte Daten vorverarbeitet und augmentiert. Darauffolgend wurden die einzelnen neuronalen Netze implementiert, angepasst, trainiert und daraufhin evaluiert. Basierend auf den Evaluationsergebnissen, wurden die Netze hinsichtlich einer Regression der Picklänge der Implantate miteinander verglichen. 

\mypar Aus der Bewertung der neuronalen Netze und dem anschließenden Vergleich hat das VGG19 die besten Ergebnisse in Bezug auf die durchschnittliche Abweichung erzielt. Diese beträgt 1,76 Pixel oder 0,077 Millimeter, sodass der Fehler bei der kleinsten und der größten Picklänge des Datensatzes 4,4 {\%} bzw. 0,8 {\%} beträgt. Überdies beträgt der Fehler des MobileNetV3 large 0,16 Pixel mehr, doch die erforderliche Rechenzeit, um die Länge eines Picks zu berechnen, wird mehr als halbiert. Hinsichtlich einer zuverlässigen und zugleich zeitlich effizienten Überprüfung der Stentgeometrie kann dieser Faktor im weiteren Verlauf des Projekts von großer Bedeutung sein.

\mypar Die in Abschnitt \ref{sec:bewertung-system-sec} erläuterten Zielanforderungen an das System sind unter den aktuellen Umständen nicht erfüllbar. Gründe dafür sind der menschliche Fehler bei der Erstellung der Labels für die Bilddaten, die verfügbare Rechenleistung und die Auflösung der aktuellen zugeschnittenen Aufnahmen.

\mypar Im nächsten Schritt des Projektes ist die Durchführung einer Hyperparameteroptimierung mit den besten faltenden neuronalen Netzen sinnvoll. Daraus können die Ergebnisse dieser Arbeit weiterhin verbessert werden. Aufgrund des verfügbaren Rechners und der begrenzten Zeit für dessen Verwendung war eine solche Optimierung der Hyperparameter nicht möglich.

\mypar Außerdem kann der Datensatz mit weiteren Flechtwinkeln erweitert und diversifiziert werden, da dieser aktuell aus Bildern mit nur fünf unterschiedlichen Flechtwinkeln besteht. Dadurch bilden sich unter den Ausgaben der trainierten Netzwerkarchitekturen fünf unterschiedliche Cluster.

\mypar Zudem können Bilder mit unterschiedlichen Lichtverhältnissen und Hintergründen hinzugefügt werden, um die Fähigkeit der Generalisierung von den neuronalen Netzen zu steigern und somit deren Leistung zu verbessern, sodass die Längen der einzelnen Picks genauer und zuverlässiger bestimmt werden können. Für die Umsetzung der angesetzten Ideen bezüglich des Datensatzes kann die Erzeugung künstlicher Bilddaten inklusive Labels untersucht werden. Zu diesem Zweck können beispielsweise Generative Adversarial Networks \cite{goodfellow2014generative}, Generative Teaching Networks \cite{such2020generative} oder weitere Verfahren erprobt werden.