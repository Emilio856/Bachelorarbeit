Die Sicherung der Qualität bei Stents ist von großer Bedeutung sowohl für den Hersteller, als auch für den Patienten. Hierbei ist der Vorgang der visuellen Inspektion aufwendig und fehleranfällig, weshalb der Vorgang automatisiert werden soll, um die Reproduzierbarkeit der Prüfung zu verbessern und ein genaueres Prüfergebnis zu erzielen. Als grundlegende Qualitätsmerkmale gelten in der Geometrie von Stents der Flechtwinkel und die Picklänge.

\mypar In der vorliegenden Arbeit wurden unterschiedliche faltende neuronale Netze implementiert und trainiert, um eine Regression der betrachteten Picklänge im Stent durchzuführen. Diese wurden darauffolgend getestet, sodass die Ergebnisse bewertet und verglichen werden konnten. Hierfür war die Erstellung eines Datensatzes, dessen Erweiterung durch Augmentation und die Vorverarbeitung der Bilddaten notwendig.

\mypar Durch die Evaluation konnte ein neuronales Netz gewonnen werden, das in der Lage ist, die Länge des mittleren Picks eines Bildes mit einer Genauigkeit von 1,764 Pixel oder 0,077 Millimeter innerhalb von 18,9 Millisekunden zu bestimmen. Im Anschluss der Arbeit ist es möglich, eine vereinfachte Form der Fehlerlokalisierung durchzuführen. 

\cleardoublepage

\chapter*{Urheberrecht}

ARM\TReg, AMBA\TReg, AXI\TTra, Cortex\TTra, TrustZone\TTra, SecurCore\TTra  , DSTREAM\TTra und weitere im Text erwähnte ARM-Produkte sowie die entsprechenden Logos sind Marken oder eingetragene Marken der Advanced RISC Machines Ltd.\par
\vspace{0.5cm}
Xilinx\TReg, Zynq\TTra und weitere im Text erwähnte Xilinx-Produkte sowie die entsprechenden Logos sind Marken oder eingetragene Marken der Xilinx Inc.\par
\vspace{0.5cm}